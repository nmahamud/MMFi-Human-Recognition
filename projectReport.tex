\documentclass[sigconf,authorversion,nonacm,11pt]{acmart}

\usepackage{booktabs} % For formal tables


\begin{document}
\title{CSE 570 Project Report}




\author{Nikhil Shah, Akshar Thakor, Nazif Mahamud}


% The default list of authors is too long for headers}
\renewcommand{\shortauthors}{N. Shah, A. Thakor, N. Mahamud}


\begin{abstract}
    This is a project that uses mmWave radar for action recognition of humans. 
\end{abstract}



\maketitle

\section{Introduction}
Identifying actions that humans perform is always a challenging thing a computer has to do. But it enables us to use the computer in different ways, often without even touching anything. 
Existing works have focused on identifying actions in a completely empty room with only a single person in view. With our project, we should be closer to recognizing the actions even though the radar is picking up other information.

\section{Related Work}

What is/are the current state of the art and what are the limitations to current approaches?


\section{Methodology}
Describe in detail what you're doing. The level of detail should be sufficient for someone to implement your work after reading this section.

\section{Dataset and Experiments}
The dataset we used is called MM-Fi: Multi-Modal Non-Intrusive 4D Human Dataset. It contains mmWave radar data, RGB videos, depth videos, and skeleton data of 40 subjects performing 27 different actions. 
There were 4 different environments. Over 11,000 samples were collected across all subjects and actions where each complete action is represented over multiple frames.

We pre-processed the data to remove the outliers and normalized the data by padding the data. After this, we kept 22 activities with a total of 10,500 samples.

\section{Evaluation}

\subsection{Metrics}
What are the metrics being used to evaluate your system and why? Examples: True Positive Rate, Accuracy, F-1 Score, Balanced Accuracy, etc.

\subsection{Baselines}
What are you comparing your system with? A baseline can be a simple straw solution or an existing approach.

\subsection{Results}
This is the most important part of the report. Please include graphs to present your results and explain them.

\section{Conclusion}
Describe the impact your project will have within the field, community, and wider audience.


\bibliographystyle{ACM-Reference-Format}
\bibliography{sample-bibliography} 

\end{document}
